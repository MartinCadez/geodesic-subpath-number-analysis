\documentclass[a4paper,12pt]{article}
\usepackage[slovene]{babel}
\usepackage[T1]{fontenc}
\usepackage[utf8]{inputenc}
\usepackage{amsmath, amssymb}
\usepackage{enumitem}
\usepackage{parskip}
\usepackage{titling}

\newtheorem{definition}{Definicija}

\title{Število geodetskih podpoti: ekstremalni grafi}
\newcommand{\subtitle}[1]{
    \posttitle{
        \par\large #1 \end{center}
    }
}
\subtitle{Kratko poročilo projekta pri predmetu Finančni praktikum}
\author{Nina Smole, Martin Čadež}
\date{9. november 2025}

\setlength{\droptitle}{-4em}

\begin{document}

\maketitle

\section{Uvod in definicije pojmov}

Cilj projekta je preučiti ekstremalne grafe glede na število geodetskih podpoti v različnih razredih grafov. Specifično nas zanima, kateri grafi dosežejo največjo možno vrednost $\text{gpn}(G)$ v naslednjih razredih povezanih grafov na $n$ vozliščih:

\begin{itemize}
    \item Dvodelni grafi
    \item Grafi brez trikotnikov
    \item Kubični grafi
    \item Vsi grafi
\end{itemize}

Za lažje nadaljevanje, najprej definiramo osnovne pojme in koncepte uporabljene za reševanje problema.

\begin{definition}
    Za povezan graf $G$ definiramo \emph{število podpoti} kot število vseh poti v grafu, vključno s trivialnimi potmi dolžine 0. Tu definiramo \emph{pot} v grafu $G$ dolžine $\ell$ kot zaporedje vozlišč $(v_0, v_1, \ldots, v_\ell)$ brez ponovitev (tj. $v_i \neq v_j$ za vse $0 \leq i < j \leq \ell$) tako, da je vsak par zaporednih vozlišč povezan z povezavo grafa $G$ (tj. $v_{i-1}v_i$ je povezava grafa $G$ za vse $1 \leq i \leq \ell$).
\end{definition}

\begin{definition}
    \emph{Število geodetskih podpoti} $\mathrm{gpn}(G)$ definiramo s štetjem le geodetskih poti. Torej, za povezan graf $G$ definiramo $\mathrm{gpn}(G)$ kot število vseh najkrajših poti v grafu, vključno s trivialnimi potmi dolžine 0.
\end{definition}

Invarianta $\mathrm{gpn}(G)$  je definirana za povezane grafe. Opazimo, da doseže svoj minimum pri drevesih. Posebaj za vsako drevo $T$ na $n$ vozliščih velja:
\[
\mathrm{gpn}(T) = \binom{n}{2}.
\]

\begin{definition}
Graf $G$ je \emph{ekstremalen} glede na dano lastnost $P$ in razred grafov $\mathcal{C}$, če med vsemi grafi iz razreda $\mathcal{C}$ doseže največjo ali najmanjšo možno vrednost lastnosti $P$.
\end{definition}

V našem primeru se sprašujemo po ekstremalnih grafih ki maksimizirajo lastnost števila geodetskih podpoti $\mathrm{gpn}(G)$ v različnih razredih povezanih grafov na $n$ vozliščih.

\begin{definition}
    Graf $G = (V, E)$ je \emph{dvodelen}, če obstaja particija množice vozlišč $V = A \cup B$, kjer $A \cap B = \emptyset$, tako da za vsako povezavo $uv \in E$ velja $(u \in A \land v \in B) \lor (u \in B \land v \in A).$
\end{definition}

\begin{definition}
    Graf $G$ je \emph{brez trikotnikov}, če ne vsebuje nobenega cikla dolžine 3 kot podgrafa.
\end{definition}

\begin{definition}
    \emph{Kubični graf} je graf, v katerem je vsako vozlišče stopnje tri. Kubični graf je 3-regularen graf.
\end{definition}

\section{Načrt dela}

Reševanja problema se bomo lotili na naslednji način:

V prvem koraku bomo izvedli analizo grafov vseh navedenih razredov za majhna $(<10)$ števila vozlišč $n$. To nam bo omogočilo boljši vpogled v obnašanje grafov. V drugem koraku bomo na podlagi prejšnje analize postavili hipotezo o tem, kateri grafi so optimalni za poljubno število vozlišč $n$. V tretjem koraku bomo hipoteze preverili z uporabo stohastičnih metod, kot je simulirano žarjenje, za večja števila vozlišč.

Ta pristop nam bo omogočil, da sistematično preučimo lastnosti števila geodetskih podpoti in identificiramo grafe, ki maksimizirajo to invarianto v različnih strukturnih razredih.



\end{document}