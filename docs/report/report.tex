\documentclass[12pt]{report}

\usepackage[letterpaper, margin=1in]{geometry}
\usepackage{microtype}
\usepackage{import}
\usepackage{algorithm}
\usepackage{algpseudocode}

\import{preamble/}{init.tex}
\usepackage[
  backend=biber,
  style=authoryear,
  natbib=true,
  maxbibnames=99,
  giveninits=true
]{biblatex}

\addbibresource{report.bib}

\hbadness=100000

\begin{document}
\thispagestyle{empty}
\customtitlepage{Število geodetskih podpoti}{Mentorja: Riste Škrekovski, Timotej Hrga }{Nina Smole, Martin Čadež}{2025}{\shortstack{Poročilo \\ Projekta}}
\newpage

\section{UVOD}

Namen raziskave je bil sistematično preučiti ekstremalne grafe glede na
število geodetskih podpoti (angl. \emph{geodesic path number},
okrajšano $gpn$) za različne strukturne razrede povezanih grafov.
Specifično nas je zanimala identifikacija grafov, ki v danem razredu dosežejo
maksimalno možno vrednost $gpn$ za fiksno število vozlišč $n$. 

Reševanja problema smo se lotili v treh glavnih korakih.
V prvem koraku smo izvedli analizo števila geodetskih podpoti
vseh enostavnih povezanih grafov z manj kot desetimi vozlišči.
V drugem koraku smo na podlagi analize postavili hipotezo o tem,
kateri grafi so optimalni za merilo dano $gpn$.
V zadnjem koraku smo za grafe z večjim število vozlišč postavljeno hipotezo 
preverili z uporabo hevrističnih metod.

\section{TEORETIČNO OZADJE}
Za natančno obravnavo problema si najprej poglejmo naslednje definicije:

\begin{definition}{Pot grafa}
  Naj bo $G$ povezan, enostaven graf. Pot dolžine $\ell$, je zaporedje
  vozlišč $(v_0, v_1, \dots, v_\ell)$ brez ponovitev, kjer je vsak par
  zaporednih vozlišč povezan s povezavo grafa $G$, tj:
  $$
  \forall \, i \in \{1, 2, \ldots, \ell \}: v_{i-1}v_i \in E(G) 
  $$
  Trivialna pot dolžine $0$ vsebuje eno vozlišče.
\end{definition}

\begin{definition}{Geodetska pot grafa}
Naj bo $G$ povezan, enostaven graf, in naj bosta $u, v \in V(G)$
dve vozlišči grafa.  Geodetska pot med $u$ in $v$ je pot 
$(v_0, v_1, \dots, v_\ell)$, ki ima minimalno možno dolžino med vsemi potmi,
ki povezujejo $u$ in $v$. 
\end{definition}

Dolžino geodetske poti označimo z $d_G(u,v)$ in jo imenujemo 
\emph{razdalja} med $u$ in $v$.

\begin{definition}{Število geodetskih podpoti}
 Naj bo $G$ povezan, enostaven graf. Število geodetskih podpoti, označeno
  $gpn(G)$, je definirano kot število vseh geodetskih (najkrajših) poti v 
  grafu, vključno s trivialnimi potmi:
  $$
  \text{gpn}(G) = \sum_{u, v \, \in \, V(G)} \text{število geodetskih poti med } u \text{ in } v
  $$
\end{definition}
  Torej za vsak par različnih vozlišč povezanega grafa obstaja vsaj ena geodetska pot, 
  ter za vsako vozlišče obstaja natanko ena trivialna pot.
\newpage

V raziskavi smo se osredotočili na naslednje strukturne razrede povezanih grafov:

\begin{definition}{Dvodelni graf}
Graf $G = (V, E)$ je \emph{dvodelen}, če obstaja razcep (particija) množice vozlišč
$V = A \cup B$,  $A \cap B = \emptyset$, tako da za vsako povezavo
$uv \in E(G)$ velja:
$$
\forall u, v \in V(G): (u \in A \land v \in B) \lor (u \in B \land v \in A)
$$
Torej, vse povezave povezujejo vozlišča iz različnih particij.
\end{definition}

\begin{definition}{Graf brez trikotnikov}
  Graf $G$ je \emph{brez trikotnikov} (angl. triangle-free graph), če v njem ne obstaja cikel dolžine $3$,
  torej če graf $G$ ne vsebuje podgrafa, izomorfnega ciklu $C_3$.
\end{definition}

\begin{claim}{Dvodelni grafi so brez trikotnikov}
Vsak dvodelni graf je hkrati graf brez trikotnikov. Obratno ne velja.
\end{claim}

\begin{definition}{Kubičen graf}
  Graf $G(V, E)$ je \emph{kubičen} (3-regularen), če ima vsako vozlišče stopnjo $3$:
  $$
  \forall \, v \in V(G): deg(G) = 3
  $$
  Torej vsako vozlišče je sosednje natanko trem vozliščem. Število povezav je enako: 
  $$ 
  |E(G)| = \frac{3 |V(G)|}{2}
  $$
\end{definition}

\begin{claim}{Obstoj kubičnih grafov}
Kubični grafi lahko obstajajo le na sodem številu vozlišč. 
\end{claim}

\begin{definition}{Ekstremalen graf}
Naj bo $\mathcal{G}$ razred grafov in $P(G)$ neka invarianta (neodvisna lastnost
grafa). Graf $G^{*}$ je ekstremalen glede na $P$, če velja:
$$
G^{*} = \arg\max_{G \in \mathcal{G}} P(G)
$$
ko želimo maksimizirati $P$ oziroma:
$$
G^{*} = \arg\min_{G \in \mathcal{G}} P(G)
$$
ko želimo minimizirati $P$.
\end{definition}

\newpage

\section{DEFINICIJA PROBLEMA}
V naši študiji smo iskali grafe, ki maksimizirajo invarianto $gpn(G)$ znotraj
razreda enostavnih povezanih označenih (angl. \textit{labeled}) grafov na
fiksnem številu vozlišč. Slednji problem lahko definiramo z naslednjim 
celoštevilskim linearnim programom:
% \begin{mdframed}[backgroundcolor=color_blue_1!10, linecolor=color_blue_2, linewidth=2pt, roundcorner=10pt, innertopmargin=10pt, innerbottommargin=10pt, skipabove=5pt, skipbelow=5pt]
%   \textbf{Celoštevilski Linearni Program za maksimizacijo geodetskega števila:} \\[3pt]
%   Naj bo $ G \in \mathcal{G}(V,E)$ neusmerjen neutežen povezan enostaven označen graf
%   nad $n$-timi vozlišči. Označimo $V(G) = \{ 1, \ldots n \}$ in 
%   $E(G) \subseteq \{ e_{ij} \mid \forall \, i,j \in V(G): e_{ij} = (i,j) \}$. \\
%   Najti želimo graf z največjim geodetskim številom.
%
%   ciljna funkcija: 
%   $ \textit{max} 
%   \left( 
%     \,\sum\limits_{i = 1}^{n} 
%     \, \sum\limits_{j = 1}^{n}
%     y_{ij}
%   \right)
%   $ \\
%   kjer :
%   \begin{enumerate}
%     \item definiramo vse matrike sosednosti za enostavne in povezane grafe
%       \begin{itemize}
%         \item $
%           \forall \, i,j \in \{1, \ldots n \}: \quad
%           x_{ij} =
%           \begin{cases}
%             1 & \! ; \, \, \,  e_{ij} \in E(G) \\
%             0 & \! ; \, \, \, sicer
%           \end{cases}
%           $ \quad (sestavimo matriko)
%         \item 
%           $
%           \sum\limits_{i < j} x_{ij} \geq n - 1
%           $ \quad
%           (graf je povezan)
%         \item
%           $
%           \forall \, i, j \in \{1, \ldots, n \}: x_{ij} = x_{ji}
%           $ \quad
%           (graf je neusmerjen)
%         \item
%           $
%           \forall i \in \{1, \ldots , n \}: x_{ii} = 0
%           $ \quad
%           (brez zank)
%         \item
%           $
%           \forall \, i ,j \in \{1, \ldots, n \} : x_{ij} \in \{0, 1\}
%           $ \quad 
%           (brez večkratnih povezav)
%       \end{itemize}
%     \item definiramo funkcijsko odvisno spremenljivko od matrike sosednosti
%       \begin{itemize}
%         \item 
%           $
%           \forall \, i,j \in \{1, \ldots , n \}: \, \,
%           y_{ij} = \{\text{število najkrajših poti med vozliščema $i$ in $j$ }$\}
%         \item 
%           $
%           \forall i,j \in \{ 1, \ldots n\}, i < j: y_{ij} = 0
%           $ \quad
%           (vsako pot štejemo le enkrat)
%         \item 
%           $
%           \forall i \in \{ 1, \ldots n\}: y_{ii} = 1
%           $ \quad
%           (trivialne poti)
%         \item 
%           $
%           \forall \, i,j \in \{1, \ldots , n \}: \, \,
%           y_{ij} \in  \mathbb{N} \cup \{0\}
%           $
%       \end{itemize}
%   \end{enumerate}

%   \vspace{0.2cm}
%   Če je podan CLP dopusten za izbrano število vozlišč $n$ in je
%   $
%   (x_{ij}^*)_{i,j = 1}^n
%   $
%   optimalna rešitev, ki predstavlja graf $G^*$, je potem to rešitev problema.
%   Pri tem je potrebno omeniti, da lahko za dani $n$ obstaja več
%   neizomorfnih grafov, ki dosežejo optimum.
%
% \end{mdframed}
% Ekspliciten primer reševanja programa je med prilogami. V programu smo
% predpostavili, da poznamo število najkrajših poti med poljubnima vozliščema,
% kar pa v splošnem ni trivialen problem, ki ga lahko rešimo v konstatnem času. 
% Potreben algoritem za izračun števila najkrajših poti, ter invariante $gpn$
% je prav tako med prilogami.

\newpage

\section{METODOLOGIJA REŠEVANJA PROBLEMA}
Problema smo se lotili na sledeč način:

\textbf{Razvojno okolje} (angl. \textit{development environment}): \\
Ker je iskalni prostor (angl. search space) vseh enostavnih povezanih označenih grafov izjemno
velik, število takih grafov za naraščajoče število vozlišč sledi zaporedju:
$$
1, \,\, 1, \,\, 2, \,\, 6, \,\, 21, \,\, 112, \,\, 853, \,\, 11117, \,\, 261080, \,\, 11716571, \, \ldots
$$
smo se naloge seveda lotili s pomočjo programskih jezikov in računalniške moči.
Zaradi številnih
sofisticiranih implementacij programske opreme za delo z grafi, smo se odločili,
da uporabimo razvojno okolje \textbf{SageMath}. Slednjega smo vspostavili 
s sistemom Docker, saj je namestitev neposredno na določene operacijske sisteme
kot je Windows brez WSL-ja (Windows Subsystem for Linux) nepraktična.
Torej, s pomočjo Docker-ja smo vzpostavili virtualni vsebnik (angl. container),
v katerem smo izvedli celotno analizo. Z virtualizacijo smo si zagotivili 
izolacijo okolja od gostiteljevega operacijskega sistema in enotno
delovno okolje na različnih platformah. Artifakt Docker-jeve slike lahko
najdete \href{https://github.com/MartinCadez/geodesic-subpath-number-analysis/blob/master/Dockerfile}{\textit{tukaj}}.

\textbf{Generacija podatkov}: \\
V prvem eksperimentalnem koraku smo definirati objekte iskalnega razreda. 
Te smo za manjše grafe lokalno generirali sami, za večje grafe pa smo se poslužili
podatkovne zbirke \textit{Bruce D. McKay za grafe}, katere repozitorij je dostopen
\href{https://users.cecs.anu.edu.au/~bdm/data/graphs.html}{\textit{tukaj}}.

V večji meri smo uporabili SageMath-ov modul \textit{nauty}, natančneje vmesnik
\textit{nauty\_geng} programa \textbf{geng}, za učinkovito generiranje neizomorfnih grafov
z določenimi lastnostmi. V prvotni fazi smo se osredotočili predvsem na tri
strukturne razrede: kubične grafe, dvodelne grafe in grafe brez trikotnikov. 
Za vsakega izmed teh razredov smo v programskem jeziku Python implementirali ustrezne razrede 
(\textit{OOP class}),
ki omogočajo generiranje vseh možnih grafov na determinističnem številu vozlišč.
Omembe vredno je, da vsak razred vključuje lastnosti grafov ter možnost
vizualizacije posameznega objekta iz razreda.
Za enostavnejšo implementacijo smo uporabili knjižnici
\textit{networkx} in \textit{matplotlib}.
Izvorna koda se nahaja \href{https://github.com/MartinCadez/geodesic-subpath-number-analysis/blob/master/src/objects.py}{\textit{tukaj}}.

Podrobneje, lokalno smo generirali vse neizomorfne enostavne povezane označene 
grafe na do vključno devetih vozliščih. Za grafe z desetimi vozlišči, smo zaradi omejenih računalniških resursov,
podatke naložili. Pri zgoraj omenjenih specifičnih strukturnih razredih, nismo
imeli težav z generacijo, saj so te skupine bistveno manjše (za grafe $|V(G)| \geq 10$).
Izvorna koda je dostopna \href{https://github.com/MartinCadez/geodesic-subpath-number-analysis/blob/master/src/analysis/data_generation.ipynb}{\textit{tukaj}}.

Vsak graf smo enkodirali z \textbf{graph6} identifikacijskim ključem, saj 
ta enolično določa vsak enostaven graf znotraj našega iskalnega prostora.

\textbf{Izračun invariante}: \\
Algoritem za izračun števila geodetskih podpoti je priložen med prilogami, 
izvorna koda pa se nahaja 
\href{https://github.com/MartinCadez/geodesic-subpath-number-analysis/blob/master/src/utils.py}{\textit{tukaj}}.

Podatke smo shranili v datoteke formata \texttt{CSV} (Comma Separated Values), 
ki smo jih po potrebi stisnili zaradi omejitev velikosti datotek na GitHub-u. 
Pri tem smo uporabili poznano knjižnico za delo s podatki \textit{pandas}
ter interaktivno računalniško okolje \textit{Jupyter Notebook}.

\newpage
Za olajšan pristop do podatkovnih zbirk so spodaj priložene povezave:
\begin{itemize}
  \item \href{https://github.com/MartinCadez/geodesic-subpath-number-analysis/blob/master/data/generated/all_graphs_9n.csv}{\textit{link}}: zbirka vseh enostavnih povezanih grafov na do vključno devetih vozliščih
  \item \href{https://github.com/MartinCadez/geodesic-subpath-number-analysis/blob/dev/data/generated/gpn_values_for10n.zip}{\textit{link}}: zbirka vseh enostavnih povezanih grafov na desetih vozliščih
  \item \href{https://github.com/MartinCadez/geodesic-subpath-number-analysis/blob/dev/data/generated/gpn_class_data.csv}{\textit{link}}: zbirka vseh grafov iz zgoraj omenjenih specifičnih strukturnih razredov
  \item \href{https://github.com/MartinCadez/geodesic-subpath-number-analysis/blob/dev/data/generated/bp_graph_data_n11.csv}{\textit{link}}: zbirka vseh dvodelnih grafov na 11 vozliščih
  \item \href{https://github.com/MartinCadez/geodesic-subpath-number-analysis/blob/dev/data/generated/bp_graph_data_n12.csv}{\textit{link}}: zbirka vseh dvodelnih grafov na 12 vozliščih
\end{itemize}

\textbf{Prečno iskanje} (angl. \textit{exhaustive search}): \\
Na podlagi prvih treh zgoraj opisanih podatkovnih zbirk smo iz razredov 
izločili zgolj \textit{ekstremalne grafe} glede na invarianto $gpn$.
Pri obdelavi in filtriranju podatkov smo ponovno uporabili knjižnico \textit{pandas}
ter sodobnejšo knjižnico \textit{polars}, razvito v okviru rustovega ekosistema.
Glavna prednost slednje je podpora lenega nalaganja
(angl. \textit{lazy loading}), ki omogoča učinkovito filtriranje velikih CSV-datotek
brez nepotrebne porabe virov.

V spodnji tabeli so zbrani ekstremalni grafi za posamezen razred glede na
število vozlišč:

\begin{center}
\begin{tcolorbox}[
  tab,
  width=\dimexpr\textwidth-1.69cm\relax,
  tabularx={c | c | c | c | c | c | c}
  ]
  \rowcolor{color_blue_2}

  graph6 ID & $|V(G)|$ & $|E(G)|$ & brez trikotnikov & dvodelen & kubičen & $gpn(G)$ \\\hline

  \textit{@} & 1 & 0 & ja & ja & ne & 1 \\\hline
  \textit{A\_} & 2 & 1 & ja & ja & ne & 3 \\\hline
  \textit{BW} & 3 & 2 & ja & ja & ne & 6 \\\hline
  \textit{Bw} & 3 & 3 & ne & ne & ne & 6 \\\hline
  \textit{C]} & 4 & 4 & ja & ja & ne & 12 \\\hline
  \textit{DFw} & 5 & 6 & ja & ja & ne & 20 \\\hline
  \textit{EFz\_} & 6 & 9 & ja & ja & ja & 33 \\\hline
  \textit{F?zv\_} & 7 & 11 & ja & ja & ne & 49 \\\hline
  \textit{F?v\_} & 7 & 12 & ja & ja & ne & 49 \\\hline
  \textit{G?zVf\_} & 8 & 14 & ja & ja & ne & 74 \\\hline
  \textit{G?zvf\_} & 8 & 15 & ja & ja & ne & 74 \\\hline
  \textit{H?BvUrw} & 9 & 17 & ja & ja & ne & 105 \\\hline
  \textit{H?BvVrw} & 9 & 18 & ja & ja & ne & 105 \\\hline
  \textit{H?ovfbo} & 9 & 16 & ja & ja & ne & 105 \\\hline
\textit{I?BvUqw\}?} & 10 & 21 & ja & ja & ne & 151 \\\hline
\end{tcolorbox}
\end{center}

Prikazi zgornjih grafov so priloženi v prilogi. Na podlagi tabele
lahko podamo hipotezo:

\begin{hypothesis}{Ekstremalni grafi za $gpn$}
  Za razred enostavnih povezanih grafov nad $n$-timi vozlišči
  so ekstremalni grafi dvodelni.
\end{hypothesis}

Slednjo lahko potrdimo že z diagramoma v prilogi, ki sta bila ustvarjena
v okolju programskega jezika R z uporabo paketov iz skupine \textit{tidyverse}
in paketa \textit{ggplot2} za vizualizacije. Izvorna koda se nahaja
\href{https://github.com/MartinCadez/geodesic-subpath-number-analysis/blob/master/src/analysis/results_leq_10/analysis_nodes.r}{\textit{tukaj}}.

\newpage

Na podlagi hipoteze, smo prečno preiskali (angl. \textit{brute-force search}) vse
dvodelne grafe na enajstih in dvanajstih vozliščih in dobili naslednje
rezultate o ekstremalnih grafih:
\begin{center}
\begin{tcolorbox}[
  tab,
  width=\dimexpr\textwidth-0.85cm\relax,
  tabularx={c | c | c | c | c | c | c}
  ]
  \rowcolor{color_blue_2}

  graph6 ID & $|V(G)|$ & $|E(G)|$ & brez trikotnikov & dvodelen & kubičen & $gpn(G)$ \\\hline

  \textit{J??E@w\{\}Fo?} & 11 & 20 & ja & ja & ne & 209 \\\hline
  \textit{K??E@w\{\}Fo\^{}?} & 12 & 25 & ja & ja & ne & 303 \\\hline

\end{tcolorbox}
\end{center}

Z dodatnim opazovanjem strukture optimalnih grafov, si lahko še dodatno skrčimo iskalni 
prostor, saj opazimo da so grafi sistematično dvodelni:
\begin{hypothesis}{Ekstremalni grafi za $gpn$}
  Največje število geodetskih podpoti za razred enostavnih povezanih
  grafov nad $n$-timi vozlišči dosežejo
  \textbf{uravnoteženi dvodelni} in precej regularni grafi, kar pomeni da
   je particija vozlišč čim bolj enakomerna:
  \begin{itemize}
    \item n je sod $\implies$ optimalna particija: $(k, k)$
    \item n je lih $\implies$ optimalna particija: $(k, k+1)$
  \end{itemize}
\end{hypothesis}

\textbf{Hevristične metode iskanja}: \\
Na podlagi zgornje hipoteze smo implementirali razred, ki generira naključen
uravnotežen dvodelen graf, pri katerem lahko stohastično uravnavamo
pričakovano gostoto povezav. Izvorna koda se nahaja
\href{https://github.com/MartinCadez/geodesic-subpath-number-analysis/blob/master/src/objects.py}{\textit{tukaj}}.

Ker se je iskalni prostor znatno zmanjšal, smo za iskanje ekstremalnih večjih 
grafov od 11 do vključno 30 vozlišči uporabili metodo naključnega iskanja
(angl. \textit{random search}). Izvorna koda je dostopna \href{https://github.com/MartinCadez/geodesic-subpath-number-analysis/blob/master/src/analysis/eda.ipynb}{\textit{tukaj}}.
Hevristični rezultati
se nahajajo \href{https://github.com/MartinCadez/geodesic-subpath-number-analysis/blob/master/data/generated/heuristic_opt_gpn_larger_nodes.csv}{\textit{tukaj}}.

Za nadaljnjo optimizacijo smo uporabili knjižnico \textit{simanneal},
ki poenostavi implementacijo simuliranega žarjenja (angl. \textit{simulated annealing, SA}).
Začetno stanje smo inicializirali z optimalnim grafom iz prejšnjega koraka. Vsako naslednje stanje smo dobili
z majhno perturbacijo grafa, kjer smo z enakomerno porazdeljeno verjetnostjo odstranili, 
dodali ali spremenili povezavo. 
Izboljšani rezultati so dostopni \href{https://github.com/MartinCadez/geodesic-subpath-number-analysis/blob/master/data/generated/sa_opt_larger_nodes.csv}{\textit{tukaj}}.

\textbf{Testiranje hipoteze}: \\
Hipotezo smo preverili s Studentovim t-testom za dva neodvisna vzorca 
(angl. \textit{Student’s two-sample t-test}) s pomočjo knjižnice \textit{scipy} z
uporabo modula \textit{stats}. Za prvi vzorec smo izbrali grafe iz
populacije vseh enostavnih povezanih grafov, za drugi pa le grafe, ki
ustrezajo razredu, opisanemu v hipotezi. Ničelna hipoteza je predpostavljala,
da je povprečje $gpn$ vrednosti uravnoteženih dvodelnih grafov večje od
povprečja $gpn$ vrednosti grafov iz splošne populacije.
Z uporabo 5-odstotne stopnje napake smo hipotezo potrdili.
Izvorna koda je na voljo \href{https://github.com/MartinCadez/geodesic-subpath-number-analysis/blob/master/src/analysis/eda.ipynb}{\textit{tukaj}}.

\newpage

\textbf{\textit{Priloga s primeroma reševanja celoštevilskega linearnega programa}}:

Za $|V|=3$ velja: \\
Obstajata le dve dopustni matriki sosednosti:
$$
\text{$X_{P_3}$ =} 
\quad
\begin{bmatrix}
  0 & 1 & 0 \\ 
  1 & 0 & 1 \\ 
  0 & 1 & 0 \\ 
\end{bmatrix}
\quad
\text{in}
\quad
\text{$X_{K_3}$ =}
\quad
\begin{bmatrix}
  0 & 1 & 1 \\ 
  1 & 0 & 1 \\ 
  1 & 1 & 0 \\ 
\end{bmatrix}
$$
potem so izpeljane matrike najkrajših poti enake:
$$
\text{$Y_{P_3}$ =} 
\quad
\begin{bmatrix}
  1 & 0 & 0 \\ 
  1 & 1 & 0 \\ 
  1 & 1 & 1 \\ 
\end{bmatrix}
\quad
\text{in}
\quad
\text{$Y_{K_3}$ =}
\quad
\begin{bmatrix}
  1 & 0 & 0 \\ 
  1 & 1 & 0 \\ 
  1 & 1 & 1 \\ 
\end{bmatrix}
$$
in so nato vrednosti objektivne funkcije enake: 
$$
gpn(P_3) = gpn(K_3) = \sum_{i=1}^3 \sum_{j=1}^3 y_{ij} = 6
$$

Za $|V|= 4$ velja: \\
Obstaja 6 dopustih matrik sosednosti za vse neizomorfne grafe:
$$
X_{P_4} =
\begin{bmatrix}
0 & 1 & 0 & 0\\ 
1 & 0 & 1 & 0\\ 
0 & 1 & 0 & 1\\ 
0 & 0 & 1 & 0\\ 
\end{bmatrix},
\quad
X_{S_4} =
\begin{bmatrix}
0 & 1 & 1 & 1\\ 
1 & 0 & 0 & 0\\ 
1 & 0 & 0 & 0\\ 
1 & 0 & 0 & 0\\ 
\end{bmatrix},
\quad
X_{C_4} =
\begin{bmatrix}
0 & 1 & 0 & 1\\ 
1 & 0 & 1 & 0\\ 
0 & 1 & 0 & 1\\ 
1 & 0 & 1 & 0\\ 
\end{bmatrix}
$$

$$
\!\!\!
\!\!\!
X_{\text{diamant}} =
\begin{bmatrix}
0 & 1 & 1 & 0\\ 
1 & 0 & 1 & 0\\ 
1 & 1 & 0 & 1\\ 
0 & 0 & 1 & 0\\ 
\end{bmatrix},
\quad
X_{K_4^-} =
\begin{bmatrix}
0 & 0 & 1 & 1\\ 
0 & 0 & 1 & 1\\ 
1 & 1 & 0 & 1\\ 
1 & 1 & 1 & 0\\ 
\end{bmatrix},
\quad
X_{K_4} =
\begin{bmatrix}
0 & 1 & 1 & 1\\ 
1 & 0 & 1 & 1\\ 
1 & 1 & 0 & 1\\ 
1 & 1 & 1 & 0\\ 
\end{bmatrix}
$$
potem so izpeljane matrike najkrajših poti enake:
$$
Y_{P_4} =
\begin{bmatrix}
1 & 0 & 0 & 0\\ 
1 & 1 & 0 & 0\\ 
1 & 1 & 1 & 0\\ 
1 & 1 & 1 & 1\\ 
\end{bmatrix},
\quad
Y_{S_4} =
\begin{bmatrix}
1 & 0 & 0 & 0\\ 
1 & 1 & 0 & 0\\ 
1 & 1 & 1 & 0\\ 
1 & 1 & 1 & 1\\ 
\end{bmatrix},
\quad
Y_{C_4} =
\begin{bmatrix}
1 & 0 & 0 & 0\\ 
1 & 1 & 0 & 0\\ 
2 & 1 & 1 & 0\\ 
1 & 2 & 1 & 1\\ 
\end{bmatrix}
$$

$$
\!\!\!
\!\!\!
Y_{\text{diamant}} =
\begin{bmatrix}
1 & 0 & 0 & 0\\ 
1 & 1 & 0 & 0\\ 
1 & 1 & 1 & 0\\ 
1 & 1 & 1 & 1\\ 
\end{bmatrix},
\quad
Y_{K_4^-} =
\begin{bmatrix}
1 & 0 & 0 & 0\\ 
2 & 1 & 0 & 0\\ 
1 & 1 & 1 & 0\\ 
1 & 1 & 1 & 1\\ 
\end{bmatrix},
\quad
Y_{K_4} =
\begin{bmatrix}
1 & 0 & 0 & 0\\ 
1 & 1 & 0 & 0\\ 
1 & 1 & 1 & 0\\ 
1 & 1 & 1 & 1\\ 
\end{bmatrix}
$$
in so nato vrednosti objektivne funkcije enake:
$$
gpn(P_4) = gpn(S_4) = gpn(K_{\text{diamant}}) = gpn(K_4) = 10, \quad
gpn(K_4^{-}) = 11, \quad
gpn(K_4) = 12
$$

% \newpage
%
% \textbf{\textit{Priloga z algoritmom $gpn$}}:

% \begin{algorithm}
% \caption{Štetje vseh najkrajših poti v grafu (GPN)}
% \label{alg:gpn}
%
% \begin{algorithmic}
% \Procedure{$gpn$}{$G = (V, E)$}
%     \State $gpn \gets |V(G)|$
%
%     \For{ $\{u, v\} \in \binom{V(G)}{2}$}
%         \State $gpn \gets gpn + 
%         \text{stevilo_najkrajsih\_poti(u,v)}$
%     \EndFor
%
%     \State \Return $gpn$
% \EndProcedure
% \end{algorithmic}
% \end{algorithm}

\newpage

\textit{\textbf{Priloga z ekstremalnimi grafi}}:

\begin{figure}[htbp!]
\centering

\begin{tabularx}{\textwidth}{@{}XXXX@{}}

    \includegraphics[width=0.23\textwidth]{../figs/opt_graphs/@.png} &
    \includegraphics[width=0.23\textwidth]{../figs/opt_graphs/A_.png} &
    \includegraphics[width=0.23\textwidth]{../figs/opt_graphs/Bw.png} &
    \includegraphics[width=0.23\textwidth]{../figs/opt_graphs/BW.png} \\

    \includegraphics[width=0.23\textwidth]{../figs/opt_graphs/C].png} &
    \includegraphics[width=0.23\textwidth]{../figs/opt_graphs/DFw.png} & 
    \includegraphics[width=0.23\textwidth]{../figs/opt_graphs/EFz_.png} & 
    \includegraphics[width=0.23\textwidth]{../figs/opt_graphs/F?zv_.png} \\

    \includegraphics[width=0.23\textwidth]{../figs/opt_graphs/F?~v_.png} &
    \includegraphics[width=0.23\textwidth]{../figs/opt_graphs/G?zVf_.png} &
    \includegraphics[width=0.23\textwidth]{../figs/opt_graphs/G?zvf_.png} &
    \includegraphics[width=0.23\textwidth]{../figs/opt_graphs/H?BvUrw.png} \\

    \includegraphics[width=0.23\textwidth]{../figs/opt_graphs/H?ovfbo.png} &
    \includegraphics[width=0.23\textwidth]{../figs/opt_graphs/H?BvVrw.png} &
    \includegraphics[width=0.23\textwidth]{../figs/opt_graphs/opt_10n.png} & 
    \includegraphics[width=0.23\textwidth]{../figs/opt_graphs/opt_graph_11.png} \\

    \includegraphics[width=0.23\textwidth]{../figs/opt_graphs/opt_graph_12.png} & 
\end{tabularx}
\end{figure}

\newpage

\textit{\textbf{Priloga z diagrami statistične analize}}:

Za vse obravnavane razrede grafov smo opazili monotono naraščanje števila 
geodetskih podpoti s številom vozlišč, kar je dobro razvidno na črtnem diagramu:

\vspace{-8pt}

\begin{figure}[htbp]
  \centering
  \begin{tcolorbox}[colframe=color_blue_2, colback=white, coltitle=black, arc=3mm, boxrule=0.4mm, width=0.65\textwidth] 
    \includegraphics[width=\textwidth]{../figs/analysis/gpn_vs_vozlisca_10.jpg}
  \end{tcolorbox}
\end{figure}

\vspace{-13pt}

Na podlagi analize povprečnih vrednosti so se dvodelni grafi izkazali 
kot optimalni razred, sledili so jim splošni grafi,
medtem ko so kubični grafi dosegali najnižje vrednosti $gpn$ pri
enakem številu vozlišč. Ker je invarianta $gpn$ precej odvisna od števila
povezav že iz same definicije, smo v naslednjem diagramu prikazali glede 
na število povezav. Poudariti je treba metodološki vidik analize,
saj predstavljeni diagram prikazuje povprečne vrednosti $gpn$
glede na število povezav za grafe, ki so bili generirani odvisnosti od števila
vozlišč. Torej tukaj nismo obravnavali vseh možnih grafov na $m$-tih povezavah,
temveč zgolj grafe z danim številom vozlišč, razvrščenih glede na število povezav.

\vspace{-5pt}

\begin{figure}[htbp]
  \centering
  \begin{tcolorbox}[colframe=color_blue_2, colback=white, coltitle=black, arc=3mm, boxrule=0.4mm, width=0.65\textwidth] 
    \includegraphics[width=\textwidth]{../figs/analysis/gpn_vs_povezave.jpg}
  \end{tcolorbox}
\end{figure}

\vspace{-8pt}

Iz diagrama je razvidno, da se povprečna vrednost $gpn$ z večanjem števila
povezav najprej povečuje, doseže maksimum, nato pa začne upadati za splošen
razred. Na začetku dodatne povezave ustvarjajo alternativne geodetske poti,
kar vrednost $gpn$ povečuje. Vendar ko graf doseže preveliko gostoto povezav
in se približa polnemu grafu, večina razdalj med vozlišči postane dolžine ena,
kar posledično zmanjša število različnih najkrajših poti. Polni graf $K_n$,
podobno kot drevesa, dosega minimalno vrednost $gpn$, tj. 
$gpn(K_n) = \binom{n}{2} + n$. 

% \newpage
%
% \section{VIRI IN LITERATURA}
%
% \parencite{baeldung_simulated_annealing}

\end{document}

