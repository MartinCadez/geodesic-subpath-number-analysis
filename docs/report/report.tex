\documentclass[12pt]{report}

\usepackage[letterpaper, margin=1in]{geometry}
\usepackage{microtype}
\usepackage{import}
\import{preamble/}{init.tex}


\begin{document}
\begin{mdframed}[backgroundcolor=color_blue_1!10, linecolor=color_blue_2, linewidth=2pt, roundcorner=10pt, innertopmargin=10pt, innerbottommargin=10pt, skipabove=5pt, skipbelow=5pt]
  \textbf{Celoštevilski Linearni Program za maksimizacijo geodetskega števila:} \\[3pt]
  Naj bo $ G \in \mathcal{G}(V,E)$ neusmerjen neutežen povezan enostaven označen graf
  nad $n$-timi vozlišči. Označimo $V(G) = \{ 1, \ldots n \}$ in 
  $E(G) \subseteq \{ e_{ij} \mid \forall \, i,j \in V(G): e_{ij} = (i,j) \}$. \\
  Najti želimo graf z največjim geodetskim številom.

  ciljna funkcija: 
  $ \textit{max} 
  \left( 
    \,\sum\limits_{i = 1}^{n} 
    \, \sum\limits_{j = 1}^{n}
    y_{ij}
  \right)
  $ \\
  kjer :
  \begin{enumerate}
    \item definiramo vse matrike sosednosti za enostavne in povezane grafe
      \begin{itemize}
        \item $
        \forall \, i,j \in \{1, \ldots n \}: \quad
        x_{ij} =
        \begin{cases}
          1 & \! ; \, \, \,  e_{ij} \in E(G) \\
          0 & \! ; \, \, \, sicer
        \end{cases}
        $ \quad (sestavimo matriko)
      \item 
      $
      \sum\limits_{i < j} x_{ij} \geq n - 1
      $ \quad
      (graf je povezan)
    \item
      $
      \forall \, i, j \in \{1, \ldots, n \}: x_{ij} = x_{ji}
      $ \quad
      (graf je neusmerjen)
    \item
      $
      \forall i \in \{1, \ldots , n \}: x_{ii} = 0
      $ \quad
      (brez zank)
    \item
      $
      \forall \, i ,j \in \{1, \ldots, n \} : x_{ij} \in \{0, 1\}
      $ \quad 
      (brez večkratnih povezav)
      \end{itemize}
    \item definiramo funkcijsko odvisno spremenljivko od matrike sosednosti
      \begin{itemize}
        \item 
        $
        \forall \, i,j \in \{1, \ldots , n \}: \, \,
        y_{ij} = \{\text{število najkrajših poti med vozliščema $i$ in $j$ }$\}
      \item 
        $
        \forall i,j \in \{ 1, \ldots n\}, i < j: y_{ij} = 0
        $ \quad
        (vsako pot štejemo le enkrat)
      \item 
        $
        \forall i \in \{ 1, \ldots n\}: y_{ii} = 1
        $ \quad
        (trivialne poti)
      \item 
        $
        \forall \, i,j \in \{1, \ldots , n \}: \, \,
        y_{ij} \in  \mathbb{N}
        $
      \end{itemize}
  \end{enumerate}

  \vspace{0.2cm}
  Če je podan CLP dopusten za izbrano število vozlišč $n$ in je
  $
  (x_{ij}^*)_{i,j = 1}^n
  $
  optimalna resitev, ki predstavlja graf $G^*$, je potem to rešitev problema.
  Pri tem je potrebno omeniti, da lahko za dani $n$ obstaja več
  neizomorfnih grafov, ki dosežejo optimum.
  
\end{mdframed}

\newpage
\textbf{Primer za $n=3$:} \\
(morda dodamo se plote kasneje vseh 8 možnosti, ali pa samo neizormorna
grafa) \\
Obstajata le dve dopustni matriki sosednosti:
$$
\text{$X_{P_3}$ =} 
\quad
\begin{bmatrix}
  0 & 1 & 0 \\ 
  1 & 0 & 1 \\ 
  0 & 1 & 0 \\ 
\end{bmatrix}
\quad
\text{in}
\quad
\text{$X_{K_3}$ =}
\quad
\begin{bmatrix}
  0 & 1 & 1 \\ 
  1 & 0 & 1 \\ 
  1 & 1 & 0 \\ 
\end{bmatrix}
$$
ter izpeljana matriki najkrajših poti:
$$
\text{$Y_{P_3}$ =} 
\quad
\begin{bmatrix}
  1 & 0 & 0 \\ 
  1 & 1 & 0 \\ 
  1 & 1 & 1 \\ 
\end{bmatrix}
\quad
\text{in}
\quad
\text{$Y_{K_3}$ =}
\quad
\begin{bmatrix}
  1 & 0 & 0 \\ 
  1 & 1 & 0 \\ 
  1 & 1 & 1 \\ 
\end{bmatrix}
$$
in je zato 
$$
gpn(P_3) = gpn(K_3) = \sum_{i=1}^3 \sum_{j=1}^3 y_{ij} = 6
$$

\newpage

\textbf{Primer za n = 4 :} \\
(dodam grafe ce je prav CLP) \\
Obstaja 6 dopustih matrik sosednosti za neizomorne grafe:
$$
X_{P_4} =
\begin{bmatrix}
0 & 1 & 0 & 0\\ 
1 & 0 & 1 & 0\\ 
0 & 1 & 0 & 1\\ 
0 & 0 & 1 & 0\\ 
\end{bmatrix},
\quad
X_{S_4} =
\begin{bmatrix}
0 & 1 & 1 & 1\\ 
1 & 0 & 0 & 0\\ 
1 & 0 & 0 & 0\\ 
1 & 0 & 0 & 0\\ 
\end{bmatrix},
\quad
X_{C_4} =
\begin{bmatrix}
0 & 1 & 0 & 1\\ 
1 & 0 & 1 & 0\\ 
0 & 1 & 0 & 1\\ 
1 & 0 & 1 & 0\\ 
\end{bmatrix}
$$

$$
\!\!\!
\!\!\!
X_{\text{diamant}} =
\begin{bmatrix}
0 & 1 & 1 & 0\\ 
1 & 0 & 1 & 0\\ 
1 & 1 & 0 & 1\\ 
0 & 0 & 1 & 0\\ 
\end{bmatrix},
\quad
X_{K_4^-} =
\begin{bmatrix}
0 & 0 & 1 & 1\\ 
0 & 0 & 1 & 1\\ 
1 & 1 & 0 & 1\\ 
1 & 1 & 1 & 0\\ 
\end{bmatrix},
\quad
X_{K_4} =
\begin{bmatrix}
0 & 1 & 1 & 1\\ 
1 & 0 & 1 & 1\\ 
1 & 1 & 0 & 1\\ 
1 & 1 & 1 & 0\\ 
\end{bmatrix}
$$
ter izpeljana matriki najkrajših poti:
$$
Y_{P_4} =
\begin{bmatrix}
1 & 0 & 0 & 0\\ 
1 & 1 & 0 & 0\\ 
1 & 1 & 1 & 0\\ 
1 & 1 & 1 & 1\\ 
\end{bmatrix},
\quad
Y_{S_4} =
\begin{bmatrix}
1 & 0 & 0 & 0\\ 
1 & 1 & 0 & 0\\ 
1 & 1 & 1 & 0\\ 
1 & 1 & 1 & 1\\ 
\end{bmatrix},
\quad
Y_{C_4} =
\begin{bmatrix}
1 & 0 & 0 & 0\\ 
1 & 1 & 0 & 0\\ 
2 & 1 & 1 & 0\\ 
1 & 2 & 1 & 1\\ 
\end{bmatrix}
$$

$$
\!\!\!
\!\!\!
Y_{\text{diamant}} =
\begin{bmatrix}
1 & 0 & 0 & 0\\ 
1 & 1 & 0 & 0\\ 
1 & 1 & 1 & 0\\ 
1 & 1 & 1 & 1\\ 
\end{bmatrix},
\quad
Y_{K_4^-} =
\begin{bmatrix}
1 & 0 & 0 & 0\\ 
2 & 1 & 0 & 0\\ 
1 & 1 & 1 & 0\\ 
1 & 1 & 1 & 1\\ 
\end{bmatrix},
\quad
Y_{K_4} =
\begin{bmatrix}
1 & 0 & 0 & 0\\ 
1 & 1 & 0 & 0\\ 
1 & 1 & 1 & 0\\ 
1 & 1 & 1 & 1\\ 
\end{bmatrix}
$$
in so nato objektivne funkcije enake:
$$
gpn(P_4) = gpn(S_4) = gpn(K_{\text{diamant}}) = gpn(K_4) = 10, \quad
gpn(K_4^{-}) = 11, \quad
gpn(K_4) = 12
$$

\newpage
(enako kot za drevesa moremo se dokazat)
\begin{claim}{Geodetsko število polnih grafov}
  Naj bo $ K_n \in \mathcal{G}(V, E)$ poln graf z $n$-timi vozlišči, potem velja:
  $$
  gpn(K_n) = n + \binom{n}{2}
  $$
\end{claim}

% \newpage
% \section{SIMULATED ANNEALING}
% let $\mathcal{G} = \{ \text{all connected labeled simple graphs \} $ 
% want: $arg max_{G \in \mathcal{G}} gpn(G)$ \\
% search space: $ \mathcal{X} = \{ G(V, E) \in \mathcal{G} \mid |V| = n \}$ 

% Energy function: $-gpn(G)$ (objective function), function to optimise/minimize
%
% Initial candidate solution: 
% hevristic!
% Path graph on n vertecies ($P_n$) coldest start, good for SA since 
% its only up from here.
%
% Initial temperature value:
% random! (fine-tunning for this value will letter be establish)
%
% Initial energy value: (Metropolis criterion)
% $ \mathcal{P}(\text{accept}) = \exp{\frac{\Delta E}{k \cdot T}}$
%
% $
% X: \Omega \to {0, 1}
% $
%
% $ 
% X_i(\omega) =
% \begin{cases}
%   1 ;& \text{$G_{\text{nov}} $ je sprejet v iteraciji $i$ pri izidu $\omega$ } \\
%   0 ;& \text{sicer}
% \end{cases}
% $
%
% $$
% \mathbb{P}(X_i = 1 \mid \Delta E, T) = 
% \begin{cases}
%   1 ;& \Delta E_i \leq 0 \\
%   \exp{(\frac{\Delta E_i}{k \cdot T_i})} ;& \Delta E > 0 
% \end{cases}
% $$
%
%
% Compute energy of Initial temperature value:
\end{document}
